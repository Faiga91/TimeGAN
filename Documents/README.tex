% Autogenerated translation of README.md by Texpad
% To stop this file being overwritten during the typeset process, please move or remove this header

\documentclass[12pt]{book}
\usepackage{graphicx}
\usepackage{fontspec}
\usepackage[utf8]{inputenc}
\usepackage[a4paper,left=.5in,right=.5in,top=.3in,bottom=0.3in]{geometry}
\setlength\parindent{0pt}
\setlength{\parskip}{\baselineskip}
\setmainfont{Helvetica Neue}
\usepackage{hyperref}
\pagestyle{plain}
\begin{document}

\chapter*{Codebase for "Time-series Generative Adversarial Networks (TimeGAN)"}

Authors: Jinsung Yoon, Daniel Jarrett, Mihaela van der Schaar

Reference: Jinsung Yoon, Daniel Jarrett, Mihaela van der Schaar, 
"Time-series Generative Adversarial Networks," 
Neural Information Processing Systems (NeurIPS), 2019.

Paper Link: https://papers.nips.cc/paper/8789-time-series-generative-adversarial-networks

Contact: jsyoon0823@gmail.com

This directory contains implementations of TimeGAN framework for synthetic time-series data generation
using one synthetic dataset and two real-world datasets.

\begin{itemize}
\item Sine data: Synthetic
\item Stock data: https://finance.yahoo.com/quote/GOOG/history?p=GOOG
\item Energy data: http://archive.ics.uci.edu/ml/datasets/Appliances+energy+prediction
\end{itemize}

To run the pipeline for training and evaluation on TimeGAN framwork, simply run 
python3 -m main\emph{timegan.py or see jupyter-notebook tutorial of TimeGAN in tutorial}timegan.ipynb.

Note that any model architecture can be used as the generator and 
discriminator model such as RNNs or Transformers. 

\subsection*{Code explanation}

(1) data\_loading.py 
- Transform raw time-series data to preprocessed time-series data (Googld data)
- Generate Sine data

(2) Metrics directory \texttt{$<$br /$>$}
 (a) visualization\emph{metrics.py 
  \_ \_ PCA and t-SNE analysis between Original data and Synthetic data \texttt{$<$br /$>$}
(b) discriminative}metrics.py 
  \_ \_ Use Post-hoc RNN to classify Original data and Synthetic data \texttt{$<$br /$>$}
(c) predictive\_metrics.py
  \_ \_ Use Post-hoc RNN to predict one-step ahead (last feature)

(3) timegan.py
- Use original time-series data as training set to generater synthetic time-series data

(4) main\_timegan.py
- Report discriminative and predictive scores for the dataset and t-SNE and PCA analysis

(5) utils.py
- Some utility functions for metrics and timeGAN.

\subsection*{Command inputs:}

\begin{itemize}
\item data\_name: sine, stock, or energy
\item seq\_len: sequence length
\item module: gru, lstm, or lstmLN
\item hidden\_dim: hidden dimensions
\item num\_layers: number of layers
\item iterations: number of training iterations
\item batch\_size: the number of samples in each batch
\item metric\_iterations: number of iterations for metric computation
\end{itemize}

Note that network parameters should be optimized for different datasets.

\subsection*{Example command}

\texttt{shell
\$ python3 main\_timegan.py --data\_name stock --seq\_len 24 --module gru
--hidden\_dim 24 --num\_layer 3 --iteration 50000 --batch\_size 128 
--metric\_iteration 10
}

\subsection*{Outputs}

\begin{itemize}
\item ori\_data: original data
\item generated\_data: generated synthetic data
\item metric\_results: discriminative and predictive scores
\item visualization: PCA and tSNE analysis
\end{itemize}

\end{document}
